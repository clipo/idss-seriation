
\documentclass[preprint,times,authoryear,10pt]{elsarticle}

%% The amssymb package provides various useful mathematical symbols
\usepackage{amssymb,amsmath}


%%%%%%%%%% Remove the following before submission %%%%%%%%%%%%%%%%%%

\usepackage{mathspec,xltxtra,xunicode}
%\usepackage{unicode-math}
%\defaultfontfeatures{Scale=MatchLowercase}
\setmainfont[Mapping=tex-text,Numbers=OldStyle]{Times New Roman}
%\setmainfont[Ligatures=TeX,Numbers=OldStyle]{Minion Pro}
\setsansfont[Mapping=tex-text]{ITC Legacy Sans Std Medium}
\setmonofont{Bitstream Vera Sans Mono}
%\setmathfont(Digits,Latin,Greek)[Script=Math,Uppercase=Italic,Lowercase=Italic]{Minion Math Semibold}
%\setmathfont[range={\mathbfup->\mathup}]{MinionMath-Bold.otf}
%\setmathfont[range={\mathbfit->\mathit}]{MinionMath-Bold.otf}
%\setmathfont[range={\mathit->\mathit}]{MinionMath-Bold.otf}

%%%%%%%%%% Remove the above before submission %%%%%%%%%%%%%%%%%%

%% The amsthm package provides extended theorem environments
%% \usepackage{amsthm}

%% The lineno packages adds line numbers. Start line numbering with
%% \begin{linenumbers}, end it with \end{linenumbers}. Or switch it on
%% for the whole article with \linenumbers after \end{frontmatter}.
\usepackage{lineno}
\usepackage{graphicx}
\usepackage{xspace}
\usepackage{bm}
\usepackage{longtable}
\usepackage{hyphenat}
\usepackage{lipsum}
\usepackage{url}
\usepackage{outlines}
\usepackage{diss-macros}
\usepackage[section,ruled]{algorithm}
\usepackage{algorithmic}
\usepackage{boxedminipage}
\usepackage[xetex,bookmarks=true,linkcolor=blue,hyperfootnotes=false,breaklinks=true,citecolor=blue,colorlinks=true]{hyperref}
\usepackage{sistyle}
\SIthousandsep{,}

\journal{Unpublished manuscript}

% Pandoc toggle for numbering sections (defaults to be off)

% Pandoc header


\begin{document}

\begin{frontmatter}


\title{Development Roadmap and Feature List for IDSS Seriation}

\author{Carl P. Lipo}
\address{Department of Anthropology and IIRMES, 1250 Bellflower Blvd, California State University at Long Beach, Long Beach CA, 90840 USA}
\ead{Carl.Lipo@csulb.edu}
\ead[url]{http://lipolab.org}

\author{Mark E. Madsen}
\address{Department of Anthropology, Box 353100, University of Washington, Seattle WA, 98195 USA}
\ead{mark@madsenlab.org}
\ead[url]{http://madsenlab.org}



\begin{keyword}
seriation \sep algorithms
\end{keyword}


\end{frontmatter}

\section{Overall Roadmap}\label{overall-roadmap}

The IDSS seriation algorithm has a Python implementation which works and
has some optimizations for performance, but needs additional features to
support planned analyses. In particular, a series of features are needed
to support large-scale simulation modeling of cultural transmission
models ``hooked'' to seriations, to study how seriations vary given
changes to the interaction network or social learning rules.

This document represents the list of desired features.

\section{Specific Feature Requests}\label{specific-feature-requests}

\begin{enumerate}
\def\labelenumi{\arabic{enumi}.}
\itemsep1pt\parskip0pt\parsep0pt
\item
  Jackknife sensitivity analysis of solution stability -- leave-one-out
  analysis of the assemblages in an input file, automatically generating
  the selected output solutions and averaging them at the end to
  calculate the variation across resampling.
\item
  Automatic performance of the spatial clustering analysis, with
  automatic labeling of graph branches by spatial clusters.
\item
  Automatic counting and stats about the number of viable frequency
  solutions for the output type chosen.
\end{enumerate}

\section{Analysis Roadmap}\label{analysis-roadmap}

\begin{enumerate}
\def\labelenumi{\arabic{enumi}.}
\itemsep1pt\parskip0pt\parsep0pt
\item
  Sample size and duration sensitivity
\item
  What patterns beyond unimodality might be useful for sequence
  alignment?\\
\item
  What can we do by combining (analytically) the frequency and
  continuity output?
\end{enumerate}

\section{Infrastructure Enhancements}\label{infrastructure-enhancements}

\begin{enumerate}
\def\labelenumi{\arabic{enumi}.}
\itemsep1pt\parskip0pt\parsep0pt
\item
  Output modules split from main algorithm -- interface to output
  modules, so we can select sets of output processing to happen without
  doing everything since some of it is time-consuming.
\item
  Database storage of output -- useful for large analysis so that we can
  pipeline analysis without writing and reading thousands of small
  files.
\item
  Work on the cluster parallelization -- perhaps we simply do separate
  runs on different machines, but any given run is multiprocessed on a
  single node? Or do we want a more serious parallelization?\\
\item
  Web service version of the basic seriation system, which takes a small
  number of assemblages for free and does output in a zip file which is
  returned?
\item
  Explore whether KNN would give us a fast lookup table when forming
  solution sets.
\end{enumerate}

\section{Completed or In Progress}\label{completed-or-in-progress}

\begin{enumerate}
\def\labelenumi{\arabic{enumi}.}
\itemsep1pt\parskip0pt\parsep0pt
\item
  Command line argument processing has been moved out of the core module
  in a branch, not yet merged into the master branch.
\end{enumerate}


%% References with bibTeX database:

\bibliographystyle{elsarticle-harv}
\bibliography{idss-featurelist-roadmap}









\end{document}

%%
%% End of file `elsarticle-template-2-harv.tex'.
